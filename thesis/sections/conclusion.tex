%% LaTeX2e class for student theses
%% sections/conclusion.tex
%% 
%% Karlsruhe Institute of Technology
%% Institute for Program Structures and Data Organization
%% Chair for Software Design and Quality (SDQ)
%%
%% Dr.-Ing. Erik Burger
%% burger@kit.edu
%%
%% Version 1.3.5, 2020-06-26

\chapter{Conclusion}
\label{ch:Conclusion}

In this thesis we examined how modern SAT solving algorithms, like the DPLL and CDCL algorithm, that normally deal with pure CNF formulas, can be extended, in order to also process DNF and AMO constraints. For this we analyzed the structure of these constraints to find out how the unit propagation process can work efficiently. Specifically for the CDCL algorithm we analyzed the structure of the conflicts, that can occur between every constraint pair. From this analysis we were able to construct general rules, that resolve these conflicts completely. General concepts in SAT solving like clause database reduction, restarts and branching heuristics were then also extended, in order to work with the new constraints. We decided to use Luby restarts as our restart strategy, EVSIDS for our branching heuristic and a constraint database reduction based on the LBD score. 

We then performed phase transition experiments on different combinations of the constraints, in order to find out, if there even exists a phase transition and if there is a correlation between the phase transition point and difficult randomized benchmarks. The experiments showed, that there exists a phase transition for every combination of constraints. We then used these results in order to create satisfiable and unsatisfiable randomized benchmark sets, in order to assess the performance of our solver. For the performance evaluation  we used a mix of randomized and industrial benchmarks and compared several different configurations of our solver with the Java based SAT solver "$Sat4j$".

We first evaluated the performance on different types of pure DNF formulas. There it was shown that for DNF constraints with a low amount of small terms, our solver wasn't able to compete with $Sat4j$. The same is true for DNF constraints with large terms and a large amount of terms. For DNF constraints with a large amount of small terms, our solver was able to outperform $Sat4j$. This is both true for the industrial CAS benchmarks, as well as the randomized benchmarks. The most probable reason for the performance advantage that $Sat4j$ has, is the better branching heuristic, which is indicated by the lower average amount of branching decision in every benchmark.

We then also used a mix of the Satlib benchmark sets, in order to compare the performance in real world scenarios. Here our solver also wasn't able to outperform $Sat4j$.

In order to assess the performance on formulas with AMO constraints, we then created benchmark sets out of sudoku playfields. Here our solver wasn't able to outperform $Sat4j$.

All in all we can conclude that there are benchmarks, where a native support for DNF constraints and AMO constraints can be beneficial to the solving time, and in some situations the new solver was able to outperform even $Sat4j$.

\section{Future Work}

The evaluation has shown a few promising results with the industrial benchmarks, that the CAS Software AG provided, and also the benchmarks, that contained DNF constraints with a large amount of small terms. In these areas our solver was able to outperform $Sat4j$. The biggest problem of our solver is, that it can't compete with $Sat4j$ in the area of pure CNF solving. This can have several reasons, but the most probable one is, that the branching heuristic, that our solver uses, just makes worse decisions than the branching heuristic of $Sat4j$. This is indicated by the high average amount of branching decisions of our solver compared to $Sat4j$. An improvement in this area would probably speed up the solving process in general, which would then also lead to a higher solving speed, when solving the other constraints natively.

Another area of improvement is the CDCL algorithm, that works with the new constraints. The DPLL configuration of our solver was often able to outperform the CDCL configurations. Here the branching heuristic could also be the reason for this performance difference. Another idea is to examine if there are more conflicts on average, if a certain constraint type occurs in a benchmark. Then it might be necessary to use a more aggressive clause database reduction algorithm.