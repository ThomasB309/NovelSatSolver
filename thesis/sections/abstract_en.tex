%% LaTeX2e class for student theses
%% sections/abstract_en.tex
%% 
%% Karlsruhe Institute of Technology
%% Institute for Program Structures and Data Organization
%% Chair for Software Design and Quality (SDQ)
%%
%% Dr.-Ing. Erik Burger
%% burger@kit.edu
%%
%% Version 1.3.5, 2020-06-26

\Abstract

The Boolean satisfiability problem is concerned with, whether there exists a set of variable assignments, that satisfy a given Boolean formula. These problems are solved by programs called "SAT solvers". Most modern SAT solvers are only capable of solving Boolean satisfiability problems in their "conjunctive normal form" (CNF). In cooperation with the CAS Software AG this thesis therefore proposes a novel SAT solver architecture, that is capable of solving pure CNF formulas in addition to "disjunctive normal form" (DNF) constraints and "at most one" (AMO) constraints. The main goal is to examine, whether these constraints have structural advantages that can be leveraged, in order to speed up the solving process in specific benchmarks. We therefore adapted modern SAT solving concepts like branching heuristics, restarts, clause database reduction and conflict resolution to the new constraint types.