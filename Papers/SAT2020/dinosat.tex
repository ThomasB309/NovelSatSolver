
\documentclass[a4paper,UKenglish,cleveref, autoref, thm-restate]{lipics-v2021}
%This is a template for producing LIPIcs articles. 
%See lipics-v2021-authors-guidelines.pdf for further information.
%for A4 paper format use option "a4paper", for US-letter use option "letterpaper"
%for british hyphenation rules use option "UKenglish", for american hyphenation rules use option "USenglish"
%for section-numbered lemmas etc., use "numberwithinsect"
%for enabling cleveref support, use "cleveref"
%for enabling autoref support, use "autoref"
%for anonymousing the authors (e.g. for double-blind review), add "anonymous"
%for enabling thm-restate support, use "thm-restate"
%for enabling a two-column layout for the author/affilation part (only applicable for > 6 authors), use "authorcolumns"
%for producing a PDF according the PDF/A standard, add "pdfa"

%\pdfoutput=1 %uncomment to ensure pdflatex processing (mandatatory e.g. to submit to arXiv)
%\hideLIPIcs  %uncomment to remove references to LIPIcs series (logo, DOI, ...), e.g. when preparing a pre-final version to be uploaded to arXiv or another public repository

%\graphicspath{{./graphics/}}%helpful if your graphic files are in another directory

\bibliographystyle{plainurl}% the mandatory bibstyle

\title{Dinosat: A SAT Solver with Native DNF Support}

\author{Thomas Bartel}{CAS Software, Karlsruhe, Germany} {thomas.bartel@cas.de}{}{}
\author{Tom\'a\v{s} Balyo}{CAS Software, Karlsruhe, Germany} {tomas.balyo@cas.de}{}{}
\author{Markus Iser}{Karlsruhe Institute of Technology, Karlsruhe, Germany} {markus.iser@kit.de}{}{}
%mandatory, please use full name; only 1 author per \author macro; first two parameters are mandatory, other parameters can be empty. Please provide at least the name of the affiliation and the country. The full address is optional. Use additional curly braces to indicate the correct name splitting when the last name consists of multiple name parts.

\authorrunning{T. Bartel and T. Balyo and M. Iser} 
%mandatory. First: Use abbreviated first/middle names. Second (only in severe cases): Use first author plus 'et al.'

\Copyright{Thomas Bartel and Tomas Balyo and Markus Iser} 
%mandatory, please use full first names. LIPIcs license is "CC-BY";  http://creativecommons.org/licenses/by/3.0/

\ccsdesc[500]{Theory of computation~Constraint and logic programming}
%mandatory: Please choose ACM 2012 classifications from https://dl.acm.org/ccs/ccs_flat.cfm 


\keywords{Sat Solver, DNF, Product Configuration} 
%mandatory; please add comma-separated list of keywords

\category{Tool Paper} %optional, e.g. invited paper

\relatedversion{} %optional, e.g. full version hosted on arXiv, HAL, or other respository/website
%\relatedversiondetails[linktext={opt. text shown instead of the URL}, cite=DBLP:books/mk/GrayR93]{Classification (e.g. Full Version, Extended Version, Previous Version}{URL to related version} %linktext and cite are optional

%\supplement{}%optional, e.g. related research data, source code, ... hosted on a repository like zenodo, figshare, GitHub, ...
%\supplementdetails[linktext={opt. text shown instead of the URL}, cite=DBLP:books/mk/GrayR93, subcategory={Description, Subcategory}, swhid={Software Heritage Identifier}]{General Classification (e.g. Software, Dataset, Model, ...)}{URL to related version} %linktext, cite, and subcategory are optional

%\funding{(Optional) general funding statement \dots}%optional, to capture a funding statement, which applies to all authors. Please enter author specific funding statements as fifth argument of the \author macro.

%\acknowledgements{I want to thank \dots}%optional

\nolinenumbers %uncomment to disable line numbering



%Editor-only macros:: begin (do not touch as author)%%%%%%%%%%%%%%%%%%%%%%%%%%%%%%%%%%
\EventEditors{John Q. Open and Joan R. Access}
\EventNoEds{2}
\EventLongTitle{42nd Conference on Very Important Topics (CVIT 2016)}
\EventShortTitle{CVIT 2016}
\EventAcronym{CVIT}
\EventYear{2016}
\EventDate{December 24--27, 2016}
\EventLocation{Little Whinging, United Kingdom}
\EventLogo{}
\SeriesVolume{42}
\ArticleNo{23}
%%%%%%%%%%%%%%%%%%%%%%%%%%%%%%%%%%%%%%%%%%%%%%%%%%%%%%

\begin{document}

\maketitle

%TODO mandatory: add short abstract of the document
\begin{abstract}
In this paper we report our preliminary results with a new kind of SAT solver called Dinosat.
Dinosat's input is a conjunction of clauses, at-most-one constraints and disjunctive normal form (DNF) formulas. The native support for DNF formulas is motivated by the application domain of SAT based product configuration. A DNF formula can also be viewed as a generalization of a clause, i.e., a clause (disjunction of literals) is special case of a DNF formula, where each term (conjunction of literals) has exactly one literal. Similarly, we can generalize the classical resolution rule and use it to resolve two DNF formulas. Based on that, the CDCL algorithm can be modified to work with DNF formulas instead of just clauses. Using randomly generated formulas (with DNFs) we experimentally show, that in certain relevant scenarios, it is more efficient to solve these formulas with Dinosat than translate them to CNF and use a state-of-the-art SAT solver. Another contribution of this paper is identifying the phase transition points for such formulas.

\end{abstract}

\section{Introduction}
TODO Tomas

\section{Preliminaries}
A \emph{Boolean variable} has two possible values: \emph{True} and \emph{False}. A \emph{literal} is a Boolean variable (positive literal) or a negation of a Boolean variable (negative literal). A \textit{clause} is a disjunction ($\lor$) of literals and, finally, a CNF formula (or just formula) is a conjuction of clauses.
A clause with only one literal is called a \textit{unit clause}.
A positive (resp. negative) literal is satisfied if the corresponding variable is assigned the value \textit{True} (resp. \textit{False})
A clause is satisfied, if at least one of its literals is satisfied and the formula is satisfied, if all its clauses are satisfied. 

The satisfiability (SAT) problem is to determine whether a given formula has a satisfying assignment, and if so, also find it. Most complete SAT solvers are based on the DPLL algorithm~\cite{Davis.1962} and its extension the CDCL algorithm~\cite{MarquesSilva.1999,Moskewicz.2001}. 

... A clause is DNF formula where each term has exactly one literal

\section{Related Work}
TODO Markus

There are solvers that natively support XOR (Cryptominisat) or Cardinality constraints (Sat4j,...) but none for DNF???

\section{Integrating DNF Reasoning into CDCL}
TODO copy from Thomas's thesis

\section{Efficient Unit Propagation of DNF Formulas}
TODO copy from Thomas's thesis

\section{Experimental Evaluation}

\section{Conclusion}

\bibliography{references}

\end{document}
